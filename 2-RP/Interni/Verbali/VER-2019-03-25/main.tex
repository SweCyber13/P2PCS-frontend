\documentclass[a4paper, 12pt]{article}
\usepackage{style}
\usepackage{hyperref}
%colori per tablle tabu
\definecolor{tableHeader}{RGB}{211, 47, 47}
\definecolor{tableLineOne}{RGB}{245, 245, 245}
\definecolor{tableLineTwo}{RGB}{224, 224, 224}
%define header tabelle tabu
\newcommand{\tableHeaderStyle}{
    \rowfont[c]{\leavevmode\color{white}\bfseries}
    \rowcolor{tableHeader}
}


\title{Verb-2019-03-25}
\author{Cyber13}
\date{March 2019}

\begin{document}
	\begin{titlepage}
		\centering Università degli Studi di Padova
		\line(1,0){350}\\
		\vspace{1.2cm}
		\logo
		\vspace{1.0cm}
		\centering{\bfseries\LARGE Verbale interno  del 25/03/2019\\}
		\vspace{0.5cm}
		\centering{\slshape\large Gruppo Cyber13 - Progetto P2PCS\\}
		\vspace{0.5cm}
		\centering{\bfseries Informazioni sul documento \\}
		\line(1,0){240}\\
		% compilare i campi per ogni documento
		\begin{tabular}{r|l}
			{\textbf{Versione}} 			& 1.0.0\\
			{\textbf{Data Redazione}} 	& 26/03/2019\\	% aggiornare la data
			{\textbf{Responsabile}} 	& Elena Pontecchiani\\	% aggiornare la data
			{\textbf{Redazione}} 		& Andrea Casagrande\\ 
			{\textbf{Verifica}} 			& Fabio Garavello\\
			{\textbf{Approvazione}} 		& Elena Pontecchiani\\
			{\textbf{Uso}} 				& Interno\\
			{\textbf{Destinatari}} 	& Cyber13\\ & Prof. Tullio Vardanega\\ & Prof. Riccardo Cardin\\
			{\textbf{Mail di contatto}} 	& swe.cyber13@gmail.com\\
		\end{tabular}\\
	\end{titlepage}

	\newpage

\newpage
		\subfile{DiarioModifiche.tex}



	\newpage
		\tableofcontents
	    	\newpage
        	\section{Informazioni sulla riunione}
\begin{itemize}
	\item \textbf{Luogo della riunione}: LabTA, Torre Tullio Levi-Civita, Via Trieste 63, Padova;
	\item \textbf{Data della riunione}: 25 Marzo 2019;
	
	\item \textbf{Partecipanti della riunione:}
		\begin{itemize}
		    \item Bira Daniel Mirel;
		    \item Casagrande Andrea;
            \item Garavello Fabio;
            \item Pontecchiani Elena.
		\end{itemize}
\end{itemize}


	
	
	
\newpage
\section{Ordine del giorno}
Durante la riunione sono stati discussi i seguenti punti:
\begin{enumerate}
	\item Discussione sull'implementazione di alcuni casi d'uso da formalizzare nel documento \AdR .
\end{enumerate}
	

\newpage
\section{Resoconto}
Nel corso della riunione si sono stati rivisti alcuni casi d'uso già pronti. Inoltre si è discusso in merito alla stesura dei seguenti casi d'uso:	
\begin{enumerate}
    \item Registrazione;
	\item Chiusura del viaggio;
	\item Gestione dati personali relativi a un account utente;
	\item Gestione dati delle auto associati a un profilo utente.
\end{enumerate}
Di seguito sono appuntate alcune delle decisioni prese nel corso dell'incontro relativamente ad alcune funzionalità del software e della gestione dei dati al suo interno, come utile riferimento alle future fasi progettuali e di sviluppo:
\begin{itemize}
    \item \textbf{Patente}: Non è obbligatoria (neanche a livello di requisiti imposti dal proponente). Effettivamente un utente potrebbe avere un auto che vorrebbe mettere a disposizione del sistema, ma non essere patentato. Piuttosto, implementarla come parte della gamification, premiando l'utente con punti o un badge per aver caricato la propria patente nel sistema. Da riflettere se volerla rendere "obbligatoria" alla prima prenotazione di un auto (se uno vuole guidare deve avere la patente, ma di fatto può essere l'utente che condivide l'auto a chiedere di vedere la patente dell'utente che noleggia, al momento della consegna delle chiavi).
    \item \textbf{Nome utente}: univoco (verifica della non presenza in database di un utente con lo stesso nickname), diventa chiave primaria in database per identificarlo univocamente, quindi non è modificabile. Accettato anche di un solo carattere, numeri o lettere che siano, ma se di un solo carattere non sono accettati caratteri speciali. Se il nome utente è di più di un carattere, invece, posso accettare anche caratteri speciali da questo insieme { . , - , \_ } 
    \item \textbf{Email}: Univoca ma modificabile. Se l'utente in fase di registrazione inserisce una mail che risulta già associata ad un altro account, l'applicazione restituisce un errore. Se l'utente autenticato all'interno del sistema va a modificare il campo email del suo profilo e inserisce una mail che risulta già associata ad un altro account, l'applicazione restituisce un errore. Ciò nonostante, l'utente può cambiarla con un'altra se valida, quindi non può essere usata come chiave univoca di identificazione (perciò la scelta a tale scopo ricade sul nome utente).
    \item \textbf{Inserimento e modifica di campi}: Poiché l'inserimento di un campo è, di fatto, la modifica di un campo vuoto, possiamo considerare l'inserimento come modifica e non è necessario gestirlo come caso separato (che richiederebbe l'uso di interfacce/pulsanti differenti per i due casi).
    \item \textbf{Controllo correttezza campi compilati}: L'ordine di risoluzione dei campi e visualizzazione di relativi eventuali errori non è formalizzata nello studio dei casi d'uso. In fase di progettazione rifletteremo su come vogliamo gestirli. In particolare si è avanzata la proposta di risolvere i campi in ordine di visualizzazione: il primo che incontra un errore ferma l'analisi dei campi e visualizza il tipo di errore. In questo modo solo quando tutti i campi avranno un contenuto corretto e valido il processo andrà a buon fine.
\end{itemize}
\newpage
\end{document}

    