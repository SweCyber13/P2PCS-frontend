Di seguito vengono riportati tutti i requisiti individuati. Essi derivano dai casi d’uso,
dal capitolato, dagli incontri con il proponente oppure da necessità interne. Per essere
più leggibili verranno separati in tabelle a seconda della loro categoria. Di ogni requisito
verranno indicati: tipologia, priorità e provenienza.
I requisiti dovranno essere classificati per tipo e importanza utilizzando la sintassi definita nel documento \NdP.
\subsection{Classificazione dei requisiti} 
    \subsubsection{Requisiti funzionali}
    Definizione requisiti delle funzioni/caratteristiche che deve fare/avere il sistema
    
%\begin{longtable}[H]

    \taburowcolors[2] 2{tableLineOne .. tableLineTwo}
    \tabulinesep = 15pt
    \everyrow{\tabucline[.4mm  white]{}}
    
    \begin{longtabu} to \textwidth { X[l] X[l] X[l] }
        \tableHeaderStyle
        Identificatore & Descrizione & Fonti \\
        
         %REGISTRAZIONE, DA RCF001 A RCF025
         RCF001 &  Un utente può creare un account per usufruire delle funzionalità offerte & UC 2, Capitolato \\
         
         RCF002 &  La registrazione per essere effettuata richiede l'inserimento del nome da associare all'account &  UC 2.1, Verbale 2019.03.12 \\
         
         RCF003 &  La registrazione per essere effettuata richiede l'inserimento del cognome da associare all'account  &  UC 2.2, Verbale 2019.03.12 \\
         
         RCF004 &  La registrazione per essere effettuata richiede l'inserimento della mail da associare all'account  &  UC 2.3, Verbale 2019.03.12 \\
         
         RCF005 &  La registrazione per essere effettuata richiede l'inserimento della password da associare all'account  &  UC 2.4, Verbale 2019.03.12 \\
         
         RCF006 &  La registrazione per essere effettuata richiede la conferma della password precedentemente inserita da associare all'account  &  UC 2.5, Verbale 2019.03.12 \\
         
         
         RCF007 &  La registrazione per essere effettuata richiede l'inserimento dell'username da associare all'account  &  UC 2.6, Verbale 2019.03.12 \\
         
         RCF008 &  La registrazione per essere effettuata deve essere confermata &  UC 2.7 \\
         
         
         RCF009 &  Il sistema segnala errore nel caso in cui la password inserita nel campo "ridigita password" sia diversa da quella inserita nella prima digitazione &  UC 2.8 \\
         
         
         RCF010 &  Il sistema segnala errore nel caso in cui la mail fornita per la registrazione sia associata già a un altro account sulla piattaforma &  UC 2.9, Interno \\
         
          
         RCF011 &  Il sistema segnala errore nel caso in cui l'username fornito per la registrazione sia associato già a un altro account sulla piattaforma  &  UC 2.10, Interno \\
         
         
         RCF012 &  Il sistema segnala errore nel caso in cui la password fornita per la registrazione non sia nel formato corretto
         &  UC 2.11, Interno \\
         
         
         
         RCF0013 &  Il sistema segnala errore nel caso in cui la mail fornita per la registrazione non sia nel formato corretto
         &  UC 2.12, Interno \\
         
         
         RCF014 &  Il sistema segnala errore nel caso in cui la registrazione tenti di essere inviata non avendo tutti i campi compilati &  UC 2.13, Interno \\
         
         
         RDF015 &  Una volta effettuata la registrazione il sistema assegna all'utente 0 punti spesa & Interno \\
         
         
         RDF016 &  Una volta effettuata la registrazione il sistema assegna all'utente il badge "Principiante" & Interno \\
         
         
         RCF017 &  Una volta effettuata la registrazione l'utente all'interno del proprio profilo non ha associata alcuna vettura & Interno \\
         
         
         RCF018 &  Una volta effettuata la registrazione l'utente all'interno del proprio profilo non ha dati personali associati & Interno \\
         
         
          RCF019 &  Il sistema richiede un formato per la password ben preciso: deve essere di almeno sei caratteri, contenere almeno una lettera maiuscola e almeno un carattere speciale  & Interno \\
          
          
          RCF020 &  Il sistema richiede un formato per la mail ben preciso: caratteri alfa-numerici, seguiti dal carattere '@', seguito da un dominio & Interno \\
          
          
          RCF021 &  Il sistema richiede che un username all'interno della piattaforma sia unico. Non possono esistere due account con lo stesso username & Verbale 2019.03.25 \\
          
          
          RCF022 &  Il sistema richiede che la mail all'interno della piattaforma sia unica. Non possono esistere due account con la stessa mail & Verbale 2019.03.25 \\
           
           
          RCF023 &  Il sistema una volta eseguita l'operazione di registrazione assegna un unico nome, un unico cognome, un'unica password e un unico username per ogni account. Questo implica che, per esempio, un account non può avere associate due mail & Verbale 2019.03.25 \\
          
          
          RCF024 &  Il sistema una volta eseguita l'operazione di registrazione assegna zero punti al rank. & Interno \\
          
          
          RCF025 &  Il sistema una volta eseguita l'operazione di registrazione non assegna badge all'account appena creato. & Interno \\
         
         
         
         
         
         
         
         
         
         %LOGIN DA RCF026 A RCF034
         RCF026 &  Un utente dispone dell'operazione di login per accedere al proprio account all'interno della piattaforma &  UC 3, Capitolato \\
         
         
         RCF027 &  Il login per essere effettuato  richiede l'inserimento dell'username  associato al profilo a cui si vuole accedere &  UC 3.1 \\
          
          
         RCF028 &  Il login per essere effettuato  richiede l'inserimento della password  associata al profilo a cui si vuole accedere &  UC 3.2 \\
         
          
         RCF029 &  Il login per essere effettuato deve essere confermato &  UC 3.3 \\
         
         
         RCF030 &  Il sistema segnala errore nel caso in cui l'utente tenti di effettuare il login non avendo tutti i campi compilati &  UC 3.4, Interno \\
         
         
         RDF031 & Una volta effettuato il login l'utente viene indirizzato alla schermata principale &   Interno \\
           
         RCF032 &  Il login per essere effettuato deve contenere credenziali valide &  Interno \\
         
         
         RDF033 &  Il login effettuato per la prima volta fa ottenere punti rank all'utente &  Interno \\
         
         
         RDF034 &  Il login effettuato la prima volta in una giornata permette la visualizzazione di una ruota della fortuna che da la possibilità all'utente di ottenere bonus all'interno dell'applicazione &  Interno \\
         
         
         %LOGOUT: RCF035-RCF036
         RCF035 &  Un utente dispone dell'operazione di logout per uscire dal proprio account &  UC 4, Capitolato \\
         
         
         RDF036 &  Una volta eseguita correttamente l'operazione di logout l'utente viene reindirizzato dal sistema nella schermata principale &  Interno \\
         
         
         %GESTIONE DATI OPZIONALI DA RCF037 A RCF041
         RCF037 &  L'utente ha la possibilità di completare il proprio profilo (che in fase di creazione contiene solo dati obbligatori al fine della registrazione) con dati facoltativi supplementari, non richiesti in fase di registrazione. Tali informazioni sono: età, sesso, città di residenza, numero della patente, data di rilascio della patente, occupazione, collegamento al profilo facebook personale &  UC 5, Interno \\
         
         
         RCF038 &  I dati facoltativi possono essere aggiunti o aggiornati all'interno del profilo utente in qualsiasi momento &  Interno \\
         
         
         RDF039 & Il primo inserimento di dati facoltativi supplementari assegna punti bonus all'utente &  Interno \\
         
         
         RCF040 & I dati opzionali con cui un utente può completare il profilo sono: foto, sesso, età, città, collegamento con social facebook &  Interno \\
         
         
         RDF041 & Il collegamento all'account facebook fa ottenere all'utente particolari bonus all'interno della piattaforma &  Interno \\
              
            
         %GESTIONE DATI PERSONALI, DA RCF042 A RCF056 
         RCF042 & L'utente ha la possibilità di modificare alcuni dei campi obbligatori associati al proprio profilo &  UC 6, Verbale 2019.03.12 \\
         
         
         RCF043 & L'utente ha la possibilità di aggiornare la password associata al proprio profilo &  UC 6.1, Verbale 2019.03.12 \\
         
         
         RCF044 & Per aggiornare la password associata al profilo dell'utente il sistema richiede, a scopi di sicurezza, di inserire la vecchia password &  UC 6.1.1, Verbale 2019.03.12 \\
         
         
         RCF045 & Per aggiornare la password associata al profilo dell'utente il sistema richiede di inserire la nuova password due volte &  UC 2.5 \\
         
         
         RCF046 & Il sistema segnala un errore nel caso in cui l'utente tenti di aggiornare la password associata al profilo con una password uguale a quella già salvata all'interno del profilo &  UC 6.1.2, Verbale 2019.03.12 \\
         
         
         RCF047 & Il sistema segnala un errore nel caso in cui l'utente tenti di aggiornare la password associata al profilo inserendo nel campo di "Ripeti password" una password diversa da quella della prima digitazione &  UC 2.8 \\
         
         
         RCF048 & Il sistema segnala un errore nel caso in cui l'utente tenti di aggiornare la password associata al profilo inserendo una password che non rispetti il formato richiesto (precedentemente espresso) &  UC 2.11 \\
           
         
         RCF049 & L'utente ha la possibilità di aggiornare la mail associata al proprio profilo & UC 6.2, Verbale 2019.03.12 \\
         
         
         RCF050 & Il sistema segnala un errore nel caso in cui l'utente tenti di aggiornare la mail associata al profilo inserendo una mail già associata ad un altro profilo presente nella piattaforma &  UC 2.9, Verbale 2019.03.12\\
         
         
          
         RCF051 & L'utente ha la possibilità di aggiornare il nome associato al proprio profilo & UC 6.3, Verbale 2019.03.12 \\
         
         
         RCF052 & L'utente ha la possibilità di aggiornare il cognome associato al proprio profilo & UC 6.4, Verbale 2019.03.12 \\
         
         
         RCF053 & L'utente ha la possibilità di eliminare il proprio profilo & UC 6.5, Verbale 2019.03.12 \\
         
         
         RCF054 & L'utente deve confermare la modifica per effettuare l'aggiornamento sui dati & UC 6.6 \\
         
         
         RCF055 & L'utente ha la possibilità di effettuare l'aggiornamento dei propri dati obbligatori all'interno del profilo in qualsiasi momento & Verbale 2019.03.12  \\
         
         
         RDF056 & Nel caso l'utente cambi idea durante l'operazione di modifica e voglia ripristinare i dati precedenti, il sistema permette di annullare le modifiche tenendo in memoria gli ultimi dati relativi al profilo dell'utente & Verbale 2019.03.12  \\
         
         
         %GESTIONE DELLE MACCHINE DEL PROFILO UTENTE, da RCF057 A RCF098
         
         RCF057 & L'utente ha la possibilità di gestire (ovvero aggiungere, modificare o eliminare) le macchine e relativi dati associati al proprio profilo &  UC 7, Capitolato \\
         
         
         RCF058 & L'utente in particolare ha la possibilità di aggiungere auto da associare al proprio profilo. Le auto inserite all'interno di un profilo indicano le vetture che l'utente rende disponibili agli altri utenti per effettuare dei viaggi con esse &  UC 7.1, Capitolato \\
         
         
         RCF059 & L'aggiunta di una nuova auto da associare al profilo utente richiede l'inserimento della targa della stessa &  UC 7.1.1, Verbale 2019.03.25 \\
         
         
         RCF060 & L'aggiunta di una nuova auto da associare al profilo utente richiede l'inserimento della marca della stessa &  UC 7.1.2, Verbale 2019.03.25 \\
          
          
          RCF061 & L'aggiunta di una nuova auto da associare al profilo utente richiede l'inserimento del modello della stessa &  UC 7.1.3, Verbale 2019.03.25 \\
          
          
          RCF062 & L'aggiunta di una nuova auto da associare al profilo utente richiede l'inserimento dell'anno di produzione della stessa &  UC 7.1.4, Verbale 2019.03.25 \\
          
          
          RCF063 & L'aggiunta di una nuova auto da associare al profilo utente richiede l'inserimento dei cavalli motore della stessa &  UC 7.1.5, Verbale 2019.03.25 \\
          
          
          RCF064 & L'aggiunta di una nuova auto da associare al profilo utente richiede l'inserimento della cilindrata della stessa & UC 7.1.6, Verbale 2019.03.25 \\
          
          
          RCF065 & L'aggiunta di una nuova auto da associare al profilo utente richiede l'inserimento del raggio di percorrenza consentito per la stessa. Con raggio di percorrenza s'intende il numero di km che il proprietario concede di effettuare per un viaggio con la sua vettura. Nel caso in cui il proprietario non voglia porre alcun limite tale valore viene settato a infinito & UC 7.1.7, Verbale 2019.03.25 \\
          
          
          RCF066 & L'aggiunta di una nuova auto da associare al profilo utente richiede l'inserimento del chilometraggio della stessa & UC 7.1.8, Verbale 2019.03.25 \\
          
          
          RCF067 & L'aggiunta di una nuova auto da associare al profilo utente richiede l'inserimento di un calendario di disponibilità della stessa, che specifica in che giorni l'utente mette a disposizione l'auto per gli altri utenti & UC 7.1.9\\
          
          
          RCF068 & L'aggiunta di una nuova auto da associare al profilo utente richiede l'inserimento di una tariffa oraria per la stessa & UC 7.1.10\\
          
          
          RCF069 & Il sistema segnala errore nel caso in cui un utente tenti di inserire una nuova vettura lasciando almeno uno dei campi sopracitati (targa, modello, marca, anno, cavalli, cilindrata, raggio e chilometraggio) vuoti & UC 7.1.11, Verbale 2019.03.25 \\
          
          
          RCF070 & Il sistema segnala errore nel caso in cui un utente tenti di inserire una nuova vettura compilando il campo targa con dati non nel formato corretto. Il formato richiesto per una targa sono caratteri alfanumerici e un numero di caratteri minimi uguale a 6 & UC 7.1.12, Verbale 2019.03.25 \\
          
          
          RCF071 & Il sistema segnala errore nel caso in cui un utente tenti di inserire una nuova vettura con un numero di targa uguale a quello già associato ad un altro profilo & UC 7.1.13, Verbale 2019.03.25 \\
          
          
          RCF072 & Il sistema segnala errore nel caso in cui un utente tenti di inserire una nuova vettura compilando in modo errato (ovvero con valori numerici minori di zero) almeno uno dei seguenti campi: anno, cavalli, cilindrata, raggio, chilometraggio & UC 7.1.14 \\
          
          
          RCF073 & L'utente una volta inserita un'auto associata al suo profilo successivamente ha la possibilità di modificarne i dati. &  UC 7.2, Verbale 2019.03.25 \\
          
          
          RCF074 & Per effettuare l'operazione di modifica dell'auto il sistema richiede prima di selezionare l'auto della quale si intendono modificare i dati &  UC 7.2.1,  \\
          
          
          RCF075 & Per effettuare l'operazione di modifica dell'auto il sistema richiede la conferma della modifica &  UC 7.2.2,  \\
          
          
          RCF076 & Il sistema permette di modificare la targa di una macchina precedentemente inserita nell'account &  UC 7.2.3, Verbale 2019.03.25 \\
          
          
          RCF077 & Il sistema permette di modificare la marca di una macchina precedentemente inserita nell'account &  UC 7.2.4, Verbale 2019.03.25 \\
          
          
          RCF078 & Il sistema permette di modificare il modello di una macchina precedentemente inserita nell'account &  UC 7.2.5, Verbale 2019.03.25 \\
          
          
          RCF079 & Il sistema permette di modificare l'anno di produzione di una macchina precedentemente inserita nell'account &  UC 7.2.6, Verbale 2019.03.25 \\
          
          
          RCF080 & Il sistema permette di modificare i cavalli motore di una macchina precedentemente inserita nell'account &  UC 7.2.7, Verbale 2019.03.25 \\
          
          
          RCF081 & Il sistema permette di modificare  la cilindrata del motore di una macchina precedentemente inserita nell'account &  UC 7.2.8, Verbale 2019.03.25 \\
          
          
          RCF082 & Il sistema permette di modificare il raggio di percorrenza consentito di una macchina precedentemente inserita nell'account &  UC 7.2.9, Verbale 2019.03.25 \\
          
          
         RCF083 & Il sistema permette di modificare il chilometraggio di una macchina precedentemente inserita nell'account &  UC 7.2.10, Verbale 2019.03.25 \\
         
         RCF084 & Il sistema permette di modificare il calendario di disponibilità di una macchina precedentemente inserita nell'account &  UC 7.2.11, Verbale 2019.03.25 \\
         
         
         RCF085 & Il sistema permette di modificare la tariffa oraria di una macchina precedentemente inserita nell'account &  UC 7.2.12, Verbale 2019.03.25 \\
         
         
         RCF086 & Il sistema segnala errore nel caso in cui l'utente tenti di modificare la targa di una vettura con un numero di targa in un formato non consentito  &  UC 7.1.12, Verbale 2019.03.25 \\
         
         
         RCF087 & Il sistema segnala errore nel caso in cui l'utente tenti di modificare la targa di una vettura con un numero di targa già esistente all'interno della piattaforma  &  UC 7.1.13, Verbale 2019.03.25 \\
         
         
         RCF088 & Il sistema segnala errore nel caso in cui l'utente tenti di modificare uno o più tra i campi anno, cilindrata, cavalli, raggio o chilometraggio con dati numerici minori di zero  &  UC 7.1.14, Verbale 2019.03.25 \\
         
         
         RCF089 & L'utente  ha la possibilità di eliminare un'auto precedentemente aggiunta all'interno del proprio profilo. Eliminando l'auto dall'account l'utente esprime la volontà che essa non possa più venire utilizzata per effettuare viaggi da parte di altri utenti &  UC 7.3, Verbale 2019.03.25 \\
         
         
         RCF090 & Il sistema permette ad ogni utente in possesso di un account di associare una o più autovettore per il proprio profilo &   Verbale 2019.03.25 \\
         
         
         RCF091 & Il sistema permette di associare esclusivamente autovetture all'interno di un profilo. Questo significa che mezzi come camion, camper, roulotte non sono consentiti  &   Verbale 2019.03.25 \\
         
         
         RCF092 & Il sistema assegna un bonus in termini di punti ad un utente che inserisce la sua prima autovettura all'interno del profilo &   Verbale 2019.03.25 \\
         
         
         RCF093 & Il sistema non permette all'utente di effettuare la modifica dei dati relativi ad un'auto se nel frattempo un altro utente sta effettuando un viaggio con quella vettura. L'utente che desidera effettuare la modifica dovrà attendere il termine del viaggio &   Verbale 2019.03.25 \\
         
         
         RCF094 & Il sistema permette all'utente di effettuare una modifica su un'auto da lui precedentemente associata al profilo in qualsiasi momento &   Verbale 2019.03.25 \\
         
         
         RCF095 & Il sistema permette all'utente di modificare i dati relativi a un auto solo se è stata precedentemente inserita almeno un auto &   Verbale 2019.03.25 \\
         
         
         RDF096 & Il primo inserimento di un auto attribuisce bonus all'utente &   Interno \\
         
         
         RDF097 & Il sistema attribuisce bonus all'utente che inserisce una tariffa oraria minore alla media per una delle sue auto &   Interno \\
         
         
         RDF098 & Il sistema permette all'utente di stabilire un calendario di disponibilità con ripetizione settimanale &   Interno \\
         
         
         %RICERCA AUTO, DA RCF099 A RC107
         RCF099 &  Un utente può effettuare una ricerca per visualizzare le auto che la piattaforma mette a disposizione per effettuare un viaggio & UC 8, Capitolato \\
         
         
         RCF100 &  Per effettuare una ricerca l'utente deve inserire le zone (partenza e arrivo) all'interno delle quali vuole effettuare il viaggio & UC 8.1, Capitolato \\
         
         
         RCF101 &  Per effettuare una ricerca l'utente deve inserire i giorni con l'orario (partenza e arrivo) all'interno dei quali vuole effettuare il viaggio & UC 8.2, Capitolato \\
         
         
         RCF102 &  Per effettuare una ricerca l'utente deve inserire il prezzo massimo che desidera pagare per effettuare il viaggio & UC 8.3, Capitolato \\
         
         
         RCF103 &  Per effettuare una ricerca l'utente deve inserire le caratteristiche che desidera per l'auto con cui effettuare il viaggio & UC 8.4, Interno \\
         
         
         RCF104 &  Per effettuare l'operazione di ricerca il sistema richiede la conferma da parte dell'utente  & UC 8.5, Interno \\
         
         
         RCF105 &  Il sistema segnala errore nel caso in cui l'utente tenti di eseguire una ricerca compilando i campi richiesti con un formato sbagliato & UC 8.6, Interno \\
         
         
         RCF106 &  Il sistema permette ad un utente loggato di eseguire una ricerca in qualsiasi momento & Interno \\
         
         
         RDF107 &  Il sistema visualizza i risultati di una ricerca in ordine di migliore corrispondenza & Interno \\
         
         
         %PRENOTAZIONE DA RCF0108 A RCF118
         RCF108 &  Un utente può prenotare un'auto con la quale effettuare un viaggio & UC 9, Capitolato \\
         
         
         RCF109 &  La prenotazione è effettivamente memorizzata nel sistema nel momento in cui il proprietario dell'auto conferma la prenotazione & Interno \\
         
         RDF110 &  La prenotazione non può essere effettuata se l'utente ha un viaggio in corso & Interno \\
         
         
         RDF111 &  La prenotazione può essere annullata fino a dodici ore prima dell'inizio del viaggio & Interno \\
         
         
         RCF112 &  Il sistema assegna dei bonus all'utente che effettua la prima prenotazione   & Interno \\
         
         
         RDF113 &  Il sistema permette prenotazioni di viaggi che durano più di ventiquattro ore & Interno \\
         
         
         RCF114 & Quando un utente richiede la prenotazione di un auto, il sistema notifica al proprietario di tale richiesta, presentandogli una schermata che gli permette di accettare, rifiutare o contattare l'utente che ha effettuato la richiesta & Interno \\
         
         
         RCF115 & Quando una prenotazione viene effettuata il sistema notifica al proprietario dell'auto l'avvenuta prenotazione   & Interno \\
         
         
         RDF116 &  Il sistema assegna un bonus all'utente proprietario che da in prestito l'auto ad un utente con rank basso & Interno \\
         
         
         RDF117 &  Il sistema assegna un bonus all'utente proprietario dopo un certo numero di prestiti & Interno \\
         
         
         RDF118 &  Il sistema assegna un bonus all'utente proprietario che presta auto agli utenti che desiderano percorrere lunghe tratte (maggiori di cento chilometri) & Interno \\
         
         
         %RECENSIONE VIAGGIO da rcf 119 a rcf 122
         RCF119 &  Un utente, a seguito della conclusione di un viaggio, può recensire il viaggio effettuato & UC 10, Verbale 2019.03.12 \\
         
         
         RCF120 &  Un utente può recensire il viaggio da lui effettuato dal momento in cui conclude il viaggio fino ad un massimo di quarantotto ore dopo & Interno \\
         
         
         RDF121 &  Il sistema assegna il bonus all'utente quando egli raggiunge una certa soglia di recensioni (la prima, dopo cinque, dopo venticinque, dopo cinquanta, dopo cento e dopo cinquecento) & Interno \\
         
         
         RDF122 &  Il sistema assegna il bonus all'utente quando egli raggiunge una certa soglia di recensioni positive da parte di altri utenti (la prima, dopo cinque, dopo venticinque, dopo cinquanta, dopo cento e dopo cinquecento) & Interno \\
         
         
         %CHIUSURA VIAGGIO da rcf 123 a rcf 127
         RCF123 &  Un utente che sta eseguendo un viaggio, nel momento in cui arriva a destinazione può concludere il viaggio & UC 11, Capitolato \\
         
         
         RCF124 &  La chiusura del viaggio per essere eseguita richiede la conferma da parte dell'utente & UC 11.1, Interno \\
         
         
         RDF125 &  La chiusura di un viaggio può essere eseguita almeno dopo un minuto dall'inizio previsto del viaggio & Interno \\
         
         
         RDF126 &  La chiusura di un viaggio avviene automaticamente nel caso l'utente si dimentichi di eseguirla. La chiusura automatica avviene passate le quarantotto ore dall'orario di chiusura prevista  & Interno \\
         
         
         RDF127 &  Conclusa l'operazione di chiusura del viaggio sulla piattaforma, se l'utente ha collegato il proprio account facebook, il sistema permette di condividere l'esperienza di viaggio vissuta sul social  & Interno \\
         
         %VISUALIZZAZIONE DATI ANAGRAFICI PERSONALI da rcf 128
         RCF128 &  Un utente può visualizzare i propri dati personali all'interno del proprio profilo. Sono sempre visibili i dati anagrafici obbligatori richiesti in fase di registrazione e, opzionalmente se l'utente ha deciso di inserirli, i dati opzionali &  UC 12, Interno \\
         
        %VISUALIZZAZIONE statistiche PERSONALI da rcf 129, rcf130
        
         RCF129 &  Un utente può visualizzare le statistiche personali all'interno del proprio profilo. In particolare sono visualizzabili: lo stato di avanzamento del livello, le missioni fatte/da fare/sbloccate/giornaliere e le statistiche riguardanti i movimenti eseguiti dall'utente all'interno dell'applicazione &  UC 16, Interno \\
         
         
         %VISUALIZZAZIONE DATI macchine personali rcf130
         RCF130 &  Un utente può visualizzare i propri dati relativi alle macchine associate al proprio profilo &  UC 13, Interno \\
         
         
         %VISUALIZZAZIONE itinerari eseguiti personali rcf131, 
         RCF131 &  Un utente può visualizzare gli itinerari da lui eseguiti, ovvero visualizza un riepilogo di tutti i viaggi da lui effettuati prendendo in prestito un'auto dal profilo di un altro utente della piattaforma. In particolare sono visibili gli orari (partenza o arrivo), il tragitto percorso, il tempo impiegato per il viaggio e il prezzo speso per il viaggio &  UC 14, Interno \\
         
         
         %VISUALIZZAZIONE profilo altro utente da rcf132, rcf 134
         RCF132 &  Un utente può visualizzare il profilo di un qualsiasi utente registrato nella piattaforma &  UC 15, Interno \\
         
         
         RCF133 &  L'utente che visita il profilo di un altro utente è in grado di vederne i dati anagrafici obbligatori e, se esistono, i dati opzionali da lui inseriti &  UC 15, Interno \\
         
         
         RCF134 &  L'utente che visita il profilo di un altro utente è in grado di vederne i dati relativi alle sue statistiche: punti rank, badge conseguiti e statistiche dell'utente &  UC 15, Interno \\
         
         %VISUALIZZAZIONE CLASSIFICHE da rcf135 a rcf136
          RCF135 &  L'utente è in grado di visualizzare delle classifiche: una globale basata sui punti rank degli utenti, classifica basata sulle statistiche e su base mensile &  UC 17, Interno \\ 
          
          
          RDF136 & Un utente acquisisce punti bonus ogni qualvolta che compare primo in una delle classifiche della piattaforma presenti &   Interno \\ 
          
          
          %VISUALIZZAZIONE CLASSIFICHE da rcf137
          RCF137 &  L'utente è in grado di visualizzare i buoni promozionali associati al suo profilo: utilizzati, da utilizzare & UC 18\\ 
          
          
          
          %RISCUOTI BUONO rcf 138, 139, 140
          RCF138 &  L'utente è in grado di riscuotere i buoni da lui cumulati & UC 19\\ 
          
          
          RCF139 &  L'utente per riscuotere un determinato buono deve preventivamente selezionarlo & UC 19.1\\ 
          
          
          RCF140 &  L'utente per riscuotere un determinato buono deve confermare l'operazione & UC 19.2\\
          
          
          %ESPLORA
          RDF141 &  L'utente è in grado di visualizzare una sezione denominata "Esplora" all'interno della quale sono presenti promozioni a termine in partnership con aziende & UC 20\\ 
          
          
            
    \end{longtabu}

%\caption{Prospetto economico - Analisi}  
%\end{longtable}
    
    \newpage
    
    
    \subsubsection{Requisiti di qualità}
      \taburowcolors[2] 2{tableLineOne .. tableLineTwo}
    \tabulinesep = 15pt
    \everyrow{\tabucline[.4mm  white]{}}
    
    \begin{longtabu} to \textwidth { X[l] X[l] X[l] }
        \tableHeaderStyle
        Identificatore & Descrizione & Fonti \\
         RCQ001 &  Il materiale consegnato deve essere coerente con quello le norme definite nel documento \NdP e \PdQ  &  Interno \\
         
         RCQ002 &  Vengono forniti tutorial per l'uso dell'applicazione agli utenti  &  Interno \\
          
         RCQ003 & La progettazione rispetta tutte le norme e le metriche indicate nei documenti \NdP e \PdQ.  &  Interno \\
           
         RCQ004 & Il processo di sviluppo e manutenzione del software si avvale del processo di \citgl{Continuos Integration} &  Interno \\
         
         RCQ005 & Durante il processo di analisi e progettazione si deve tener conto delle problematiche di usabilità &  Interno \\
         
         RCQ006 & Durante il processo di analisi e progettazione si deve tener conto delle problematiche di user-experience &  Interno \\
         
         RCQ007 & L'interfaccia grafica deve essere intuitiva &  Interno \\
         
         
         
    \end{longtabu}
    
    \newpage
    
    \subsubsection{Requisiti di vincolo}
    \taburowcolors[2] 2{tableLineOne .. tableLineTwo}
    \tabulinesep = 15pt
    \everyrow{\tabucline[.4mm  white]{}}
    
    \begin{longtabu} to \textwidth { X[l] X[l] X[l] }
        \tableHeaderStyle
        Identificatore & Descrizione & Fonti \\
        
        RCV001 & Il sistema deve essere un'applicazione mobile composta da front-end (quasi totalmente fornito) e back-end.  &  Capitolato \\
        
        RDV002 & La parte di back-end è composta da una parte di interazione con servizi AWS (nello specifico la parte relativa al login) &  Verbale 2019.03.12 \\
        
        RCV003 & L'utente interagisce con un'applicazione in lingua italiana &  Capitolato \\
        
        RCV004 & L'applicazione deve includere meccanismi di \citgl{Gamification} &  Capitolato \\
        
        RCV005 & Per l'implementazione di meccanismi di Gamification viene fatto riferimento al modello \citgl{Octalysis} &  Capitolato \\
           
        RCV006 &  Per lo sviluppo dell'applicazione il linguaggio di programmazione utilizzato è Kotlin & Verbale 2019.03.12 \\
         
        RDV007 &  L'applicazione deve funzionare su piattaforma Android dalla versione 6 in poi & Verbale 2019.03.12 \\
          
        RCV008 &  L'IDE di riferimento per la creazione dell'app è \citgl{Android Studio} &  Verbale 2019.03.12  \\
        
        RDV009 & Il codice verrà reso disponibile open-source tramite repository online sulla piattaforma Github &  Verbale 2019.03.12  \\
          
        RDV010 & Devono essere eseguiti i \citgl{test UI} per tutti gli scenari dell'applicazione &  Verbale 2019.03.12 \\
           
        RDV011 & Come modello di sviluppo software viene indicato il modello "test-driven-development" (TDD) &  Verbale 2019.03.12 \\
           
        RDV012 & Viene utilizzato il framework Espresso per la creazione di test dell'interfaccia utente per simulare le interazioni di un utente con l'applicazione &  Verbale 2019.03.12 \\
           
           
           %tecnologie come angularjs da mettere come desiderabili anche se front end per lo più già fornito???
           
           
           
           
         
         
    \end{longtabu}
    
    
    
    \newpage


\subsection{Tracciamento}
    \subsubsection{Tracciamento fonti-requisiti}

    \taburowcolors[2] 2{tableLineOne .. tableLineTwo}
    \tabulinesep = 15pt
    \everyrow{\tabucline[.4mm  white]{}}
    %da rivedere
    
    \begin{longtabu} {X[c] X[c]}
    \hline
    Fonte & Requisiti individuati \\ 
     {Capitolato} &  RCF001, RCF026, RCF035, RCF057, RCF058, RCF099, RCF100, RCF101, RCF108, RCF123, RCV001, RCV003, RCV004, RCV005 \\ 
                                
                               
     {Interni} & RCF010, RCF011, RCF012, RCF013, RCF014, RCF015, RCF016, RCF017, RCF018, RCF019, RCF020,
     RCF024, RCF025,
     RCF030, RCF031, RCF032, RDF033, 
     RCF036, RCF037, RCF038, RCF039, RCF040, RDF041,
     RCF096, RCF097, RDF098,
     RCF102, RCF103, RCF104, RCF105, RCF106, RDF107,
     RCF109, RDF110, RDF111, RCF112, RDF113, RCF114, RCF115, RDF116, RDF117, RDF118,
     RCF120, RDF121, RDF122,
     RCF124, RDF125, RDF126, RDF127,
     RCF128, RCF130, RCF131, RCF132, RCF133, RCF134, RCF136, RCQ001, RCQ002, RCQ003, RCQ004, RCQ005, RCQ006, RCQ007 \\  
     
     
                             
    {Verbale 2019.03.12} & RCF002, RCF003, RCF004, RCF005, RCF006, RCF007, RCF041, RCF042, RCF043, RCF045, RCF048, RCF049, RCF050, RCF051, RCF052, RCF054, RCF055, RCF119, RDV002, RCV006, RDV007, RCV008, RDV009, RDV010, RDV011, RDV012 \\                   
    
   
   
   {Verbale 2019.03.25} & RCF021, RCF022, RCF023, RCF058, RCF059, RCF060, RCF061, RCF062, RCF063, RCF064, RCF065, RCF068, RCF069, RCF070, RCF071, RCF072, RCF075, RCF076, RCF077, RCF078, RCF079, RCF080, RCF081, RCF082, RCF085, RCF086, RCF087, RCF088, RCF089, RCF090, RCF091, RCF092, RCF093, RCF094 \\ 
   
   
   {UC 2} & RCF001 \\ 
   
   {UC 2.1} & RCF002 \\ 
   
   {UC 2.2} & RCF003 \\ 
   
   {UC 2.3} & RCF004 \\ 
   
   {UC 2.4} & RCF005 \\ 
   
   {UC 2.5} & RCF006, RCF045 \\
   
   {UC 2.6} & RCF007 \\ 
   
   {UC 2.7} & RCF008 \\ 
   
   {UC 2.8} & RCF009, RCF047 \\ 
   
   {UC 2.9} & RCF010, RCF050 \\
   
   {UC 2.10} & RCF011 \\ 
   
   {UC 2.11} & RCF012, RCF048 \\
   
   {UC 2.12} & RCF013 \\ 
   
   {UC 2.13} & RCF014 \\ 
   
   {UC 3} & RCF026 \\
   
   {UC 3.1} & RCF027 \\ 
   
   {UC 3.2} & RCF028 \\ 
   
   {UC 3.3} & RCF029 \\ 
   
   {UC 3.4} & RCF030 \\
   
   {UC 4} & RCF035 \\ 
   
   {UC 5} & RCF035 \\ 
   
   {UC 6} & RCF042 \\ 
   
   {UC 6.1} & RCF043 \\ 
   
   {UC 6.1.1} & RCF044 \\ 
   
   {UC 6.1.2} & RCF046 \\ 
   
   {UC 6.2} & RCF049 \\ 
   
   {UC 6.3} & RCF051 \\
   
   {UC 6.4} & RCF052 \\ 
   
   {UC 6.5} & RCF053 \\ 
   
   {UC 6.6} & RCF054 \\ 
   
   
   
   {UC 7} & RCF057 \\ 
   
   {UC 7.1} & RCF058 \\ 
   
   {UC 7.1.1} & RCF059 \\ 
   
   {UC 7.1.2} & RCF060 \\ 
   
   {UC 7.1.3} & RCF061 \\ 
   
   {UC 7.1.4} & RCF062 \\ 
   
   {UC 7.1.5} & RCF063 \\  
   
   {UC 7.1.6} & RCF064 \\ 
   
   {UC 7.1.7} & RCF065 \\ 
   
   {UC 7.1.8} & RCF066 \\  
   
   {UC 7.1.9} & RCF067 \\ 
   
   {UC 7.1.10} & RCF068 \\ 
   
   {UC 7.1.11} & RCF069 \\ 
   
   {UC 7.1.12} & RCF070, RCF086\\ 
   
   {UC 7.1.13} & RCF071, RCF087 \\ 
   
   {UC 7.1.14} & RCF072, RCF088 \\ 
   
   {UC 7.2} & RCF073 \\ 
   
   {UC 7.2.1} & RCF074 \\ 
   
   {UC 7.2.2} & RCF075 \\ 
   
   {UC 7.2.3} & RCF076 \\ 
   
   {UC 7.2.4} & RCF077 \\ 
   
   {UC 7.2.5} & RCF078 \\ 
   
   {UC 7.2.6} & RCF079 \\ 
   
   {UC7. 2.7} & RCF080 \\ 
   
   {UC 7.2.8} & RCF081 \\ 
   
   {UC 7.2.9} & RCF082 \\ 
   
   {UC 7.2.10} & RCF083 \\
   
   {UC 7.2.11} & RCF084 \\ 
   
   {UC 7.2.12} & RCF085 \\ 
 
    {UC 7.3} & RCF089 \\ 
    
    
    
   {UC 8} & RCF099 \\ 
   
   {UC 8.1} & RCF100 \\ 
   
   {UC 8.2} & RCF101 \\ 
   
   {UC 8.3} & RCF102 \\
   
   {UC 8.4} & RCF103 \\ 
   
   {UC 8.5} & RCF104 \\ 
   
   {UC 8.6} & RCF105 \\ 
   
   
   {UC 9} & RCF108 \\
   
   {UC 9.2} & RCF109 \\
   
   {UC 10} & RCF119 \\
   
   {UC 11} & RCF123 \\
   
   {UC 11.1} & RCF124 \\
   
   {UC 12} & RCF128 \\
   
   {UC 13} & RCF130 \\
   
   {UC 14} & RCF131 \\
   
   {UC 15} & RCF132 \\
   
   {UC 16} & RCF129 \\
   
   {UC 17} & RCF135 \\
   
   {UC 18} & RCF137 \\
   
   {UC 19} & RCF138 \\
   
   {UC 19.1} & RCF139 \\
   
   {UC 19.2} & RCF140 \\
   
    {UC 20} & RCF141 \\
    
    {UC 21} & RDF034 \\
                                
    
    \end{longtabu}
    
    
    
    
    
    
    
    
    
    
    
   