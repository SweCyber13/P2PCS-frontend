Come modello di ciclo di vita del software è stato scelto dal gruppo di adottare il \citgl{modello incrementale}. Tale modello prevede che, previa la conoscenza del \citgl{sistema} che si intende implementare, lo sviluppo del progetto venga suddiviso in varie fasi. Il termine di ogni fase definisce una \citgl{milestone}. Ogni fase è a sua volta suddivisa in sotto-attività tali da poter essere affrontate come raffinamenti o estensioni delle attività concluse in precedenza, senza però impattare sul resto del progetto.
\\L'utilizzo del modello incrementale prevede quindi di suddividere il sistema che si intende realizzare in parti più piccole: andranno identificate quelle più critiche ed essenziali, le quali saranno implementate per prime. In questo modo è possibile porre una solida base di partenza: in seguito si procederà incrementalmente fino allo sviluppo del sistema completo. Necessario è porre un limite massimo ben definito al numero di fasi/incrementi, pena il rischio di trasformare il modello incrementale in un loop potenzialmente infinito di iterazioni non incrementali.
\\\\Ad esempio, nella fase di Analisi dei Requisiti del sistema che si intende realizzare è importante distinguere quali di essi siano essenziali, quindi andranno implementati per primi, e quali invece siano solo desiderabili, quindi implementabili in un secondo momento.
\\A questo fine risulta importante la comunicazione con la \citgl{proponente} \citgl{GaiaGo}, il cui massimo coinvolgimento è uno degli obiettivi della pianificazione.

\subsection{Fasi di Progetto}
Per il progetto P2PCS sono state identificate 5 fasi fondamentali. Le attività che compongono ciascuna fase sono descritte e pianificate nella sezione 4 del presente Piano di Progetto:
\begin{itemize}
    \item Analisi: La prima fase è data da tutte quelle attività necessarie ad organizzare il lavoro del team. In particolare consiste nella realizzazione di alcuni documenti ad uso interno ed esterno cui ogni membro dovrà fare riferimento durante le successive fasi di lavoro.
    \item Analisi di dettaglio: Una breve fase intermedia tra la realizzazione dell'Analisi e l'inizio della Progettazione. Necessaria per raffinare e verificare quanto prodotto durante la fase di Analisi in vista della prima consegna del materiale e per presentare il lavoro del team alla \citgl{Proponente} e al \citgl{Committente}.
    \item Progettazione della base tecnologica: Fase di definizione delle tecnologie e delle scelte dei Progettisti e conseguente realizzazione di un prototipo del software sulla base di tali scelte.
    \item Progettazione di dettaglio e codifica: In questa fase viene sviluppato il corpo principale dell'applicativo scopo del progetto. I programmatori scrivono il codice seguendo le direttive dei progettisti, i quali affinano la progettazione del programma a partire dalle basi e le tecnologie descritte nella fase precedente.
    \item Validazione e collaudo: Fase di completamento e raffinamento dell'applicazione, i verificatori si assicurano che il software sviluppato sia conforme ai requisiti individuati dagli analisti e i programmatori incrementano l'applicativo allo scopo di eliminare errori e mancanze riscontrate durante il collaudo. La documentazione, incluso un manuale per l'utilizzo dell'applicazione, vengono rifiniti e completati. Al termine della fase il prodotto finito viene consegnato dal team alla \citgl{Proponente} e al \citgl{Committente} per l'approvazione.
\end{itemize}
Le fasi di Analisi e Progettazione non sono reiterabili: una volta identificati i Requisiti, essi restano fissati. Se così non fosse, non sarebbe possibile pianificare i cicli di incremento. Possibili fattori di rischio che potrebbero causare problemi e discostare il ciclo di sviluppo come pianificato allo scopo di adattarlo al modello incrementale sono riportati e analizzati nella sezione 4 del presente Piano di Progetto.