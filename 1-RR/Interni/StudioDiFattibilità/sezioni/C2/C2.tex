\subsection{Informazioni sul capitolato}
    \begin{itemize}
        \item Nome del progetto: Colletta - piattaforma per raccolta dati mediante esercizi grammaticali
        \item \textbf{Proponente}: MIVOQ
        \item \textbf{Committente}: Prof. Tullio Vardanega, Prof. Riccardo Cardin
    \end{itemize}
\subsection{Descrizione}
L'obiettivo del capitolato è lo sviluppo di una piattaforma collaborativa di raccolta dati in cui gli utenti possano creare e/o
svolgere esercizi di grammatica (per esempio esercizi di analisi grammaticale). I  dati  raccolti  devono essere utilizzabili da sviluppatori e ricercatori al fine di insegnare ad un elaboratore a svolgere i medesimi esercizi mediante tecniche di apprendimento automatico supervisionato.
   

\subsection{Studio del dominio}
     \begin{itemize}
        \item \textbf{Dominio applicativo}: Il dominio applicativo a cui Mivoq fa riferimento è quello del commercio di strumenti di sintesi vocale che si occupano dell'analisi automatica del testo per determinare la pronuncia corretta di quest'ultimo. Questi strumenti si utilizzano nell'ambito dell'istruzione per facilitare il compito degli insegnanti e per migliorare l'apprendimento degli studenti.

        \item \textbf{Dominio tecnologico}: 
            \begin{itemize}
                \item Firebase: piattaforma per  immagazzinare i dati;  
                \item Hunpos, FreeLing: software \citgl{open source} per lo svolgimento degli esercizi grammaticali 
            \end{itemize}
        

    \end{itemize}
\subsection{Esito finale}
    \begin{itemize}
        \item Aspetti positivi: L'idea di creare un software utile per l'apprendimento automatico è stata reputata molto interessante poiché quest'ultima richiede l'applicazione di principi di \citgl{machine learning}.
        \item Fattori di rischio: 
            \begin{itemize}
                \item Le tecnologie fornite sono sconosciute ai partecipanti del gruppo, quindi il tempo necessario ad apprenderne l'uso non è stato ritenuto sufficiente;
                \item Il capitolato non è più disponibile.
            \end{itemize}
        \item Conclusioni
            \begin {itemize}
                \item Il gruppo non ha la possibilità di scegliere il capitolato;
                \item Scelta: rigettato.
            \end {itemize}
    \end{itemize}