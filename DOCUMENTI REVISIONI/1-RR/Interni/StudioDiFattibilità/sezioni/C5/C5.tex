\subsection{Informazioni sul capitolato}
    \begin{itemize}
        \item Nome del progetto:  P2PCS -"\citgl{Car sharing} \citgl{Peer to Peer}"
        \item \textbf{Proponente}: GaiaGo
        \item \textbf{Committente}: Prof Tullio Vardanega, Prof. Riccardo Cardin
    \end{itemize}
\subsection{Descrizione}
Il capitolato ha come oggetto lo sviluppo di un'applicazione \citgl{Android} a partire da librerie fornite dal proponente, che ha già maturato esperienza nell'ambito del car sharing.
Nello specifico si desidera implementare un servizio di condivisione dell'auto peer to peer che applichi principi di \citgl{gamification} basati sul \citgl{framework} \citgl{Octalysis} per incentivare l'utente all'uso dell'applicazione.

\subsection{Studio del dominio}
     \begin{itemize}
        \item \textbf{Dominio applicativo}: Il capitolato fa riferimento a un'applicazione Android rivolta ad utenti interessati alla condivisione della propria auto o a richiederne una in noleggio.
        \item \citgl{Dominio tecnologico}:
            \begin{itemize}
                \item Kotlin: Linguaggio di sviluppo suggerito dal proponente;
                \item AWS: Piattaforma \citgl{cloud} di Amazon per la gestione di registrazione e login degli utenti;
                \item Bitrise: Ambiente di test automatici basato sulla pratica della \citgl{Continuous Integration} e sulla \citgl{Test Driven Development};
                \item Espresso: Framework per l'esecuzione dei \citgl{test UI} all'interno dell'ambiente Bitrise;
                \item Android studio: \citgl{IDE di sviluppo} specifico per app \citgl{Android};
                \item API Google Maps: Strumenti che permettono di ottenere informazioni sulla posizione geografica e visualizzazione di mappe.
            \end{itemize}
    \end{itemize}
\subsection{Esito finale}
    \begin{itemize}
        \item Aspetti positivi:
            \begin{itemize}
                \item Le tecnologie scelte sono ampiamente utilizzate e quindi reputate didatticamente interessanti dai componenti del gruppo;
                \item I componenti del gruppo sono interessati ad approfondire l'aspetto della \citgl{gamification} integrata all'interno di un'applicazione;
                \item I membri del team reputano accattivante l'implementazione di meccaniche \citgl{peer to peer} per un servizio di questo tipo rispetto ad approcci standard per il \citgl{car sharing}. 
            \end{itemize}
        \item Fattori di rischio:
            \begin{itemize}
                \item I componenti del gruppo non possiedono conoscenze specifiche in merito alle piattaforme utilizzate dall'azienda, pertanto sarà necessario uno studio preparatorio. 
            \end{itemize}
        \item Conclusioni
            \begin{itemize}
                \item Alla luce dei capitolati disponibili e a quanto essi propongono il progetto in questione è stato reputato il più interessante da parte del gruppo;
		        \item Scelta: accettato.
            \end{itemize}
    \end{itemize}