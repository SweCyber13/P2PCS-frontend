\documentclass[a4paper, 12pt]{article}
\usepackage{style}


%colori per tablle tabu
\definecolor{tableHeader}{RGB}{211, 47, 47}
\definecolor{tableLineOne}{RGB}{245, 245, 245}
\definecolor{tableLineTwo}{RGB}{224, 224, 224}
%define header tabelle tabu
\newcommand{\tableHeaderStyle}{
    \rowfont[c]{\leavevmode\color{white}\bfseries}
    \rowcolor{tableHeader}
}

\title{StudioDiFattibilità}
\author{Cyber13}
\date{March 2019}

\begin{document}

	\begin{titlepage}
		\centering Università degli Studi di Padova
		\line(1,0){350}\\
		\vspace{1.2cm}
		\logo
		\vspace{1.0cm}
		\centering{\bfseries\LARGE STUDIO DI FATTIBILITÀ \\}
		\vspace{0.5cm}
		\centering{\slshape\large Gruppo Cyber13 - Progetto P2PCS\\}
		\vspace{0.5cm}
		\centering{\bfseries Informazioni sul documento \\}
		\line(1,0){240}\\
		% compilare i campi per ogni documento
		\begin{tabular}{r|l}
			{\textbf{Versione}} 			& 1.0.0\\
			{\textbf{Data Redazione}} 	& 13/03/2019\\	% aggiornare la data
			{\textbf{Responsabile}} 	& Matteo Squeri\\	% aggiornare la data
			{\textbf{Redazione}} 		& Daniel Mirel Bira\\ & Andrea Casagrande\\ & Elena Pontecchiani\\ & Ilaria Rizzo\\
			{\textbf{Verifica}}  & Fabio Garavello\\ & Matteo Squeri\\ 
			{\textbf{Approvazione}} 		& Matteo Squeri\\
			{\textbf{Uso}} 				& Interno\\
			{\textbf{Destinatari}} 	& Cyber13\\ & Prof. Tullio Vardanega\\ & Prof. Riccardo Cardin\\
			{\textbf{Mail di contatto}} 	& swe.cyber13@gmail.com \\
		\end{tabular}\\
	\end{titlepage}

	\newpage
	    \subfile{sezioni/DiarioModifiche.tex}
    \newpage
		\tableofcontents
		\newpage
            \section{Introduzione}
                \subsection{Scopo del documento}
Lo scopo di tale documento consiste nello stabilire le norme utilizzate per verificare la qualità del prodotto e del processo. A tale scopo sarà svolta una continua attività di verifica per rilevare e correggere eventuali anomalie tramite metriche di seguito descritte.
\subsection{Scopo del prodotto}
Lo scopo del prodotto è quella di realizzare un'applicazione \citgl{Android} che implementi un servizio di car sharing \citgl{peer-to-peer}.
\subsection{Glossario}
Onde evitare ambiguità o incomprensioni di natura lessicale, si allega il \G.
All'interno del documento saranno presenti parole di ambito specifico, uso raro che potrebbero creare incomprensioni. Per una maggiore leggibilità tali parole sono riconoscibili all'interno dei vari documenti in quanto scritte in corsivo e con un 'g' a pedice tra barre orizzontali (per esempio \citgl{Glossario})
\subsection{Riferimenti}
    \subsubsection{Riferimenti normativi}
    \begin{itemize}
        \item \NdP
        \item Capitolato C5: Car sharing peer to peer.
        \\ \url{ https://www.math.unipd.it/~tullio/IS-1/2018/Progetto/C5.pdf}
    \end{itemize}
    \subsubsection{Riferimenti informativi}
    \begin{itemize}
        \item Slide corso Ingegneria del Software:
        \\ \url{https://www.math.unipd.it/~tullio/IS-1/2018/Dispense/L03.pdf}
        \item Slide corso Ingegneria del Software: Qualità di prodotto
        \\ \url{https://www.math.unipd.it/~tullio/IS-1/2018/Dispense/L13.pdf}
        \item Slide corso Ingegneria del Software: Qualità di processo
        \\ \url{https://www.math.unipd.it/~tullio/IS-1/2018/Dispense/L14.pdf}
        \item Standard ISO/IEC 9126
        \\ \url{https://it.wikipedia.org/wiki/ISO/IEC_9126}
        \item Indice Gulpease:
        \\ \url{https://it.wikipedia.org/wiki/Indice_Gulpease}
        \item Framework Octalysis:
        \\ \url{https://yukaichou.com/gamification-examples/}
        \item ISO/IEC 15504:
        \\ \url{https://en.wikipedia.org/wiki/ISO/IEC_15504}
        
    \end{itemize}
    
    
                \newpage
            \section{Capitolato C1}
                \subsection{Informazioni sul capitolato}
    \begin{itemize}
        \item Nome del progetto: Butterfly
        \item \textbf{\citgl{Proponente}}: Imola Informatica
        \item \textbf{\citgl{Committente}}: Prof. Tullio Vardanega, Prof. Riccardo Cardin
    \end{itemize}
\subsection{Descrizione}
L'obiettivo del capitolato è lo sviluppo di una piattaforma software che permetta l'accentramento e la standardizzazione delle segnalazioni generate da software utilizzati nell'ambito dello sviluppo, più precisamente in quello delle pratiche di \citgl{Continuous Integration} e \citgl{Continuous Delivery}. Il problema sollevato dal proponente risiede nella necessità di accedere a molteplici piattaforme diverse per visionare tali segnalazioni, ciascuna con una propria struttura specifica e in alcuni casi con limitate capacità di configurazione. Al fine di facilitare la gestione e l'accesso alle informazioni contenute nelle segnalazioni è quindi stato proposto il progetto oggetto del capitolato. Viene proposto di realizzare, attraverso un \citgl{design pattern} di tipo \citgl{publisher-subscriber}, una serie di componenti che si interfaccino direttamente con i diversi strumenti software da cui vengono generate le segnalazioni, per poi procedere a redistribuirle agli utenti destinatari nella forma da loro desiderata, come ad esempio una e-mail.
\subsection{Studio del dominio}
     \begin{itemize}
        \item \textbf{\citgl{Dominio applicativo}}: Il capitolato fa riferimento all'ambito dello sviluppo software tramite l'utilizzo di strumenti per l'implementazione delle pratiche di \citgl{Continuous Integration} e \citgl{Continuous Delivery}. Tali strumenti producono statistiche e segnalazioni automatiche contenenti informazioni riguardanti problemi riscontrati durante lo sviluppo. Queste sono solitamente di interesse per specifici utenti, i quali hanno la necessità di consultarle in modo rapido.
        \item \textbf{\citgl{Dominio tecnologico}}:
        \begin{itemize}
        \item RedMine: Software per la pianificazione di progetti e gestione delle segnalazioni tramite \citgl{interfaccia web};
        \item GitLab: Piattaforma per la gestione di \citgl{repository} \citgl{Git};
        \item SonarQube: Strumento per l'analisi statica del codice;
        \item Telegram e Slack: Applicazioni per invio e ricezione di messaggi tra utenti, i messaggi generati dal prodotto devono poter essere inoltrati ad entrambi i software;
        \item Linguaggi Java, Python o Node.js per lo sviluppo del software;
        \item Apache Kafka: Una piattaforma \citgl{open source} utilizzabile come Broker, ovvero il componente dell'applicativo che si occupa di raccogliere le segnalazioni come messaggi e inoltrarli all'utente nel software scelto.
        \end{itemize}
    \end{itemize}
\subsection{Esito finale}
    \begin{itemize}
        \item Aspetti positivi:
        \begin{itemize}
        \item La stretta relazione del progetto con strumenti molto utilizzati nello sviluppo software come SonarQube e GitLab è stata vista come un ottima opportunità per acquisire conoscenze che potrebbero tornare utili in futuro;
        \item La possibilità di utilizzare il linguaggio Java, con cui tutti i membri del gruppo hanno già una discreta familiarità;
        \item Lo studio e la messa in pratica del design pattern; \citgl{publisher-subscriber} per i componenti del software è stato ritenuto interessante.
        \end{itemize}
        \item Fattori di rischio:
        \begin{itemize}
        \item La gestione della grande quantità di software esterni con cui si deve interfacciare l'applicativo potrebbe risultare troppo onerosa in termini di sviluppo;
        \item Alcune Tecnologie come Apache Kafka sono completamente sconosciute ai membri del gruppo. Lo studio necessario per apprenderle ed utilizzarle in modo sufficientemente efficace potrebbe richiedere troppo tempo.
        \end{itemize}
        \item Conclusioni:
        \begin{itemize}
        \item Sebbene il capitolato sia stato giudicato positivamente dai componenti del gruppo, i fattori di rischio elencati precedentemente hanno portato i membri del team a reputarlo non sostenibile;
        \item Scelta: rigettato.
        \end{itemize}
        
    \end{itemize}
                \newpage
            \section{Capitolato C2}
                \subsection{Informazioni sul capitolato}
    \begin{itemize}
        \item Nome del progetto: Colletta - piattaforma per raccolta dati mediante esercizi grammaticali
        \item \textbf{Proponente}: MIVOQ
        \item \textbf{Committente}: Prof. Tullio Vardanega, Prof. Riccardo Cardin
    \end{itemize}
\subsection{Descrizione}
L'obiettivo del capitolato è lo sviluppo di una piattaforma collaborativa di raccolta dati in cui gli utenti possano creare e/o
svolgere esercizi di grammatica (per esempio esercizi di analisi grammaticale). I  dati  raccolti  devono essere utilizzabili da sviluppatori e ricercatori al fine di insegnare ad un elaboratore a svolgere i medesimi esercizi mediante tecniche di apprendimento automatico supervisionato.
   

\subsection{Studio del dominio}
     \begin{itemize}
        \item \textbf{Dominio applicativo}: Il dominio applicativo a cui Mivoq fa riferimento è quello del commercio di strumenti di sintesi vocale che si occupano dell'analisi automatica del testo per determinare la pronuncia corretta di quest'ultimo. Questi strumenti si utilizzano nell'ambito dell'istruzione per facilitare il compito degli insegnanti e per migliorare l'apprendimento degli studenti.

        \item \textbf{Dominio tecnologico}: 
            \begin{itemize}
                \item Firebase: piattaforma per  immagazzinare i dati;  
                \item Hunpos, FreeLing: software \citgl{open source} per lo svolgimento degli esercizi grammaticali 
            \end{itemize}
        

    \end{itemize}
\subsection{Esito finale}
    \begin{itemize}
        \item Aspetti positivi: L'idea di creare un software utile per l'apprendimento automatico è stata reputata molto interessante poiché quest'ultima richiede l'applicazione di principi di \citgl{machine learning}.
        \item Fattori di rischio: 
            \begin{itemize}
                \item Le tecnologie fornite sono sconosciute ai partecipanti del gruppo, quindi il tempo necessario ad apprenderne l'uso non è stato ritenuto sufficiente;
                \item Il capitolato non è più disponibile.
            \end{itemize}
        \item Conclusioni
            \begin {itemize}
                \item Il gruppo non ha la possibilità di scegliere il capitolato;
                \item Scelta: rigettato.
            \end {itemize}
    \end{itemize}
                \newpage
            \section{Capitolato C3}
                \subsection{Informazioni sul capitolato}
    \begin{itemize}
        \item Nome del progetto: G\&B - monitoraggio intelligente di processi DevOps
        \item \textbf{Proponente}: Zucchetti
        \item \textbf{Committente}: Prof. Tullio Vardanega, Prof. Riccardo Cardin
    \end{itemize}
\subsection{Descrizione}
L'azienda che propone il capitolato presenta un prodotto \citgl{open source}, Grafana, con cui monitora i propri sistemi e da la possibilità di creare \citgl{plug-in} per personalizzarne l'uso. Il capitolato infatti richiede lo sviluppo di quest'ultimo che grazie alle \citgl{reti Bayesiane} ha lo scopo di prevedere quali potrebbero essere gli interventi da eseguire e le zone di intervento nella linea di produzione del software. La scelta dell'utilizzo delle \citgl{reti Bayesiane} è data dall'esigenza di raccogliere la competenza degli esperti in un sistema probabilistico, collegata poi ai dati effettivamente raccolti sul campo per determinare infine quali eventi non ancora presentatosi saranno più probabili. 

\subsection{Studio del dominio}
     \begin{itemize}
        \item \textbf{Dominio applicativo}: Nel dettaglio l'azienda ha lo scopo di utilizzare tale servizio per monitorare allarmi e segnalazioni tra gli operatori del servizio \citgl{Cloud} e la linea di produzione del software.
        \item \textbf{Dominio tecnologico}:
            \begin{itemize}
                \item Grafana: conoscenza dell'\citgl{ambiente di sviluppo} del \citgl{plug-in};
                \item Javascript: linguaggio su cui si basa il plug-in;
                \item \citgl{Reti Bayesiane} come modello di gestione dei dati.
            \end{itemize}
    \end{itemize}
\subsection{Esito finale}
    \begin{itemize}
        \item Aspetti positivi: Il progetto si presenta molto interessante dal punto di vista degli ambiti d'uso e dell'ambiente di sviluppo, la presentazione risulta molto dettagliata quindi si ha la chiara visione del lavoro da svolgere.
        \item Fattori di rischio:
            \begin{itemize}
                \item La documentazione dell'ambiente di sviluppo Grafana risulta poco soddisfacente per garantire un apprendimento sufficiente alla creazione dell'applicazione richiesta;
                \item Il capitolato non è più disponibile.
            \end{itemize}
        \item Conclusioni:
            \begin{itemize}
                \item Sebbene il capitolato sia stato giudicato complessivamente molto interessante dai componenti del gruppo, esso non è più disponibile;
		        \item Scelta: rigettato.
            \end{itemize}
    \end{itemize}
                \newpage
            \section{Capitolato C4}
                \subsection{Informazioni sul capitolato}
    \begin{itemize}
        \item Nome del progetto: MegAlexa
        \item \textbf{Proponente}: ZERO12
        \item \textbf{Committente}: Prof Tullio Vardanega, Prof. Riccardo Cardin
    \end{itemize}
\subsection{Descrizione}
Il capitolato ha come scopo lo sviluppo di \citgl{skill per Alexa} di Amazon in grado di avviare dei \citgl{workflow} (routine di micro-funzioni) creati dagli utenti tramite interfaccia web o
mobile app per \citgl{iOS} e \citgl{Android}. I workflow sono completamente personalizzabili dagli utenti finali, in modo tale che una volta creati possano soddisfare appieno tutte
le loro esigenze senza che ci sia la necessità di dover fare manualmente la routine ogni volta. Per poter avviare i workflow creati
basta pronunciare "Alexa, esegui la routine" dove il nome del workflow viene assegnato nel momento della sua creazione. Il workflow può eseguire le micro-funzioni
anche in modo combinato, per esempio si può utilizzare una micro-funzione che legge un \citgl{feed rss} combinata con un'altra micro-funzione che filtra le notizie del feed rss interessate.
Per fare ciò è necessario creare una piattaforma (web e mobile) che offre delle micro-funzioni che possano essere collegate tra di loro creando il workflow voluto dall'utente.


\subsection{Studio del dominio}
     \begin{itemize}
        \item \textbf{Dominio applicativo}: Il capitolato fa riferimento all'interazione con i servizi di Amazon.
        \item \textbf{Dominio tecnologico}: 
            \begin{itemize}
                \item Amazon Web Services: Un insieme di servizi di \citgl{cloud} computing che compongono la piattaforma \citgl{on demand} offerta dall'azienda Amazon;
                \item API Gateway: Servizio completamente gestito che semplifica agli sviluppatori la creazione, la pubblicazione, la manutenzione, il monitoraggio e la protezione
		        delle \citgl{API} su qualsiasi scala;
                \item Lambda: Piattaforma che consente di eseguire codice senza dover effettuare il \citgl{provisioning} né gestire server;
                \item DynamoDB: Database che supporta i modelli di dati di tipo documento e di tipo chiave-valore che offre alte prestazioni a qualsiasi livello;
                \item Node.js: Piattaforma \citgl{open source} \citgl{event-driven} per l'esecuzione di codice JavaScript Server-side;
                \item Framework Node.js Express: Web \citgl{framework} per Node.js;
                \item Twitter bootstrap: Raccolta di strumenti liberi per la creazione di siti e applicazioni per il Web. Essa contiene modelli di progettazione basati su HTML, CSS
		        e JavaScript;
		        \item Alexa developer: Piattaforma di supporto per lo sviluppo di \citgl{skill Alexa};
        \end{itemize}
    \end{itemize}
\subsection{Esito finale}
    \begin{itemize}
        \item Aspetti positivi:
            \begin{itemize}
                \item Ritenuta interessante l'idea di creare una skill per Amazon Alexa;
                \item Lo studio e l'utilizzo dei servizi Amazon è reputato molto interessante dai componenti del gruppo;
                \item L'utilizzo dei framework Node.js Express e bootstrap facilitano lo sviluppo del codice;
                \item Interessante l'utilizzo del codice lato server.
            \end{itemize}
        \item Fattori di rischio:
            \begin{itemize}
                \item La maggior parte delle tecnologie risulta sconosciuta ai componenti del gruppo, inoltre si presentano innumerevoli \citgl{casi d'uso} da gestire per poter offrire all'utente
		        un'ampia gamma di micro-funzioni.
            \end{itemize}
        \item Conclusioni
            \begin{itemize}
                \item Sebbene il capitolato sia stato giudicato complessivamente molto interessante dai componenti del gruppo, il tempo necessario per la gestione di tutti i \citgl{casi d'uso}
		        e di tutte le micro-funzioni è stato considerato non sostenibile;
                \item Scelta: rigettato.
            \end{itemize}
    \end{itemize}
                \newpage
            \section{Capitolato C5}
                \subsection{Informazioni sul capitolato}
    \begin{itemize}
        \item Nome del progetto:  P2PCS -"\citgl{Car sharing} \citgl{Peer to Peer}"
        \item \textbf{Proponente}: GaiaGo
        \item \textbf{Committente}: Prof Tullio Vardanega, Prof. Riccardo Cardin
    \end{itemize}
\subsection{Descrizione}
Il capitolato ha come oggetto lo sviluppo di un'applicazione \citgl{Android} a partire da librerie fornite dal proponente, che ha già maturato esperienza nell'ambito del car sharing.
Nello specifico si desidera implementare un servizio di condivisione dell'auto peer to peer che applichi principi di \citgl{gamification} basati sul \citgl{framework} \citgl{Octalysis} per incentivare l'utente all'uso dell'applicazione.

\subsection{Studio del dominio}
     \begin{itemize}
        \item \textbf{Dominio applicativo}: Il capitolato fa riferimento a un'applicazione Android rivolta ad utenti interessati alla condivisione della propria auto o a richiederne una in noleggio.
        \item \citgl{Dominio tecnologico}:
            \begin{itemize}
                \item Kotlin: Linguaggio di sviluppo suggerito dal proponente;
                \item AWS: Piattaforma \citgl{cloud} di Amazon per la gestione di registrazione e login degli utenti;
                \item Bitrise: Ambiente di test automatici basato sulla pratica della \citgl{Continuous Integration} e sulla \citgl{Test Driven Development};
                \item Espresso: Framework per l'esecuzione dei \citgl{test UI} all'interno dell'ambiente Bitrise;
                \item Android studio: \citgl{IDE di sviluppo} specifico per app \citgl{Android};
                \item API Google Maps: Strumenti che permettono di ottenere informazioni sulla posizione geografica e visualizzazione di mappe.
            \end{itemize}
    \end{itemize}
\subsection{Esito finale}
    \begin{itemize}
        \item Aspetti positivi:
            \begin{itemize}
                \item Le tecnologie scelte sono ampiamente utilizzate e quindi reputate didatticamente interessanti dai componenti del gruppo;
                \item I componenti del gruppo sono interessati ad approfondire l'aspetto della \citgl{gamification} integrata all'interno di un'applicazione;
                \item I membri del team reputano accattivante l'implementazione di meccaniche \citgl{peer to peer} per un servizio di questo tipo rispetto ad approcci standard per il \citgl{car sharing}. 
            \end{itemize}
        \item Fattori di rischio:
            \begin{itemize}
                \item I componenti del gruppo non possiedono conoscenze specifiche in merito alle piattaforme utilizzate dall'azienda, pertanto sarà necessario uno studio preparatorio. 
            \end{itemize}
        \item Conclusioni
            \begin{itemize}
                \item Alla luce dei capitolati disponibili e a quanto essi propongono il progetto in questione è stato reputato il più interessante da parte del gruppo;
		        \item Scelta: accettato.
            \end{itemize}
    \end{itemize}
                \newpage
            \section{Capitolato C6}
                \subsection{Informazioni sul capitolato}
    \begin{itemize}
        \item Nome del progetto: Soldino
        \item \textbf{Proponente}: Red Babel
        \item \textbf{Committente}: Prof Tullio Vardanega, Prof. Riccardo Cardin
    \end{itemize}
\subsection{Descrizione}
Il capitolato ha come scopo lo sviluppo di diverse \citgl{D-apps} (ovvero applicazioni che usano contratti intelligenti per il loro trattamento)
in esecuzione sulle EVM (Ethereum Virtual Machine). La piattaforma desiderata avrà tre attori principali, ognuno dei quali con funzionalità ben definite: 
\begin{itemize}
        \item Il governo: responsabile della coniazione e distribuzione di denaro ai cittadini e alle imprese. All'interno del sistema la valuta usata è un \citgl{token} compatibile con \citgl{ERC20}, detta Cubit. Inoltre detiene una lista di attività di tutte le aziende autorizzate ad usufruire del sistema;
	   
        \item Gli imprenditori: Le cui attività possono comprare e/o fornire 
        beni o servizi.
	    Inoltre l'azienda su base trimestrale è obbligata a pagare una tassa, IVA, al governo;

        \item I cittadini: acquistano beni e servizi dalle aziende attraverso i Cubit. Ed eventualmente se interessati possono aprire la propria attività registrandosi all'opportuno registro detenuto dal governo.
       
    \end{itemize}
L'interazione tra le varie componenti avverrà tramite un insieme di contratti intelligenti.
Pertanto il progetto sarà composto da due macro moduli:
    \begin{itemize}
        \item WEB/UI: deve contenere un insieme di pagine Web che consentano all'utente di interfacciarsi con la EVM, e quindi di eseguire le operazioni che gli sono consentite;
    
        \item Smart contract: contengono le transazioni che vengono sviluppate con un meccanismo di deposito in garanzia dalla rete Ethereum. Le transazioni vengono memorizzate in una struttura dati detta \citgl{Blockchain}, che sostanzialmente è una catena di blocchi contenenti appunto le transazioni. 
    \end{itemize}
La validazione delle singole transazioni è affidata a un meccanismo di consenso distribuito su tutti i nodi della rete Ethereum.


\subsection{Studio del dominio}
     \begin{itemize}
        \item \tetxbf{Dominio applicativo}: Il capitolato fa riferimento al contesto dell' \citgl{e-commerce} e pagamento di tasse tramite criptovaluta;
        \item \textbf{Dominio tecnologico}:
            \begin{itemize}
                \item EVM (Ethereum Virtual Machine): Piattaforma che permette la gestione della rete Ethereum, lo stato interno e il calcolo. Consente quindi 
                \item Ethereum: Piattaforma per consentire agli utenti di scrivere \citgl{D-apps} che utilizzano tecnologia \citgl{blockchain};
		        scritti in linguaggio Solidity;
		        verificare e eseguire codice sulla \citgl{blockchain}. Il codice in esecuzione sulla EVM è contenuto nei cosiddetti "contratti intelligenti", 
                \item Framework Truffle: \citgl{Framework} di sviluppo per Ethereum che ne facilita lo sviluppo. Ad esso sono delegati la maggior parte dei compiti di
		        routine.
                \item Meta Mask: \citgl{Plug-in} browser che consente di eseguire le dApp di Ethereum usando un nodo pubblico invece che eseguire un nodo Ethereum completo.
		        Include inoltre un vault di identità protetto che fornisce all'utente l'interfaccia per gestire la sua identità su diversi siti e firmare
                 \item Ropsten: Rete di test ufficiale, creata da The Ethereum Foundation;
		        transazioni Ethereum in \citgl{blockchain}; 
		         a basso costo e scalabili sulla rete Ethereum. Funziona con qualsiasi token compatibile a ERC20.
                 \item Raiden Network: Strato infrastrutturale sopra la blockchain di Ethereum. Ha lo scopo di consentire pagamenti quasi istantanei,
            \end{itemize}
    \end{itemize}
\subsection{Esito finale}
    \begin{itemize}
        \item Aspetti positivi:
             \begin{itemize}
                \item Studio della piattaforma Ethereum, reputata molto interessante dai componenti del gruppo;
                \item Studio del linguaggio orientato agli oggetti Solidity per la creazione di Smart Contracts;
                \item Il framework Truffle facilita lo sviluppo, in quanto facilita e automatizza tutte le operazioni di routine;
            \end{itemize}
        \item Fattori di rischio:
            \begin{itemize}
                \item La maggior parte delle tecnologie è completamente sconosciuta ai componenti del gruppo, e lo studio di un numero di tecnologie così alto
		        potrebbe non essere sostenibile dal gruppo.
            \end{itemize}
        \item Conclusioni
            \begin{itemize}
                \item Sebbene il capitolato sia stato giudicato complessivamente molto interessante dai componenti del gruppo, il tempo necessario per lo
		        studio dei framework utilizzati da Soldino è stato considerato non sostenibile;
		        \item Scelta: rigettato.
            \end{itemize}
    \end{itemize}
            
\end{document}
